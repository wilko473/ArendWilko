\documentclass[a4paper]{article}

\usepackage{xcolor}
\usepackage{fancyheadings}
\usepackage{listings}

\newcommand{\todo}[1]{\textcolor{red}{[#1]}}
\lhead{Open Universiteit}
\chead{IM0102, Design patterns}
\rhead{Eindopdracht}

\begin{document}
\pagestyle{fancy}

\section*{Studentgegevens}
\begin{description}
	\item [Cursuscode] IM0102
	\item \todo{Titel van het scenario}
	\item [Naam] Arend Slomp
	\item [Studentnummer] 838768442
	\item [Naam] Wilko
	\item [Studentnummer] \todo{invullen}
\end{description}

\section*{Aanpak}
\todo{<Geef aan hoe jullie de opdracht hebben aangepakt en wie wat heeft gedaan, maximaal 1 A-4. Geef expliciet aandacht aan de volgorde van activiteiten>}
We hebben een afspraak gemaakt om bij elkaar langs te gaan. In eerste instantie om te kijken wat de opdracht was om er in dezelfde sessie in detail op in te gaan. We hebben bekeken hoe de huidige opbouw was van het project in combinatie met de diagrammen die in het boek staan. Toen zijn we uitwerkingen gaan maken van hoe een oplossing er uit zou kunnen zien en een analyse gedaan van de vraag die gesteld wordt in de opdracht.


\section{Probleemanalyse}


\section{Ontwerp}


\section{Keuzen}



\section{Sourcecode}

\end{document}
