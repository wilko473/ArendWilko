\documentclass[a4paper]{article}

\usepackage{xcolor}
\usepackage{fancyhdr}
\usepackage{listings}
\usepackage[dutch]{babel}
\usepackage{graphicx}

\newcommand{\todo}[1]{\textcolor{red}{[#1]}}
\lhead{Open Universiteit}
\chead{IM0102, Design patterns}
\rhead{Eindopdracht}

\begin{document}
\pagestyle{fancy}

\section*{Studentgegevens}
\begin{description}
	\item [Cursuscode] IM0102
	\item \todo{Titel van het scenario}
	\item [Naam] Arend Slomp
	\item [Studentnummer] 838768442
	\item [Naam] Wilko
	\item [Studentnummer] \todo{invullen}
\end{description}

\section*{Aanpak}
\todo{<Geef aan hoe jullie de opdracht hebben aangepakt en wie wat heeft gedaan, maximaal 1 A-4. Geef expliciet aandacht aan de volgorde van activiteiten>}
We hebben een afspraak gemaakt om bij elkaar langs te gaan. In eerste instantie om te kijken wat de opdracht was om er in dezelfde sessie in detail op in te gaan. We hebben bekeken hoe de huidige opbouw was van het project in combinatie met de diagrammen die in het boek staan. Toen zijn we uitwerkingen gaan maken van hoe een oplossing er uit zou kunnen zien en een analyse gedaan van de vraag die gesteld wordt in de opdracht.
Aan het einde van de eerste sessie hebben we besproken beide opdracht een en twee te maken zodat we daarna een zicht krijgen op de potentiele mogelijkheden.


\section{Probleemanalyse}
Er wordt gevraagd thema's toe te voegen aan het ontwerp.
In eerste instantie moet er dan worden afgevraagd wat een thema is, welke functionaliteit het moet kunnen bevatten en welke overeenkomsten we hier zien.
In ieder geval hebben we onderscheiden dat verschillende thema's verschillende stijlen hebben. Dit betekent dat we een stijl gaan vastleggen in het thema.
Verder is er geconstateerd dat de achtergrondkleur van verschillende slides aanpasbaar moet zijn.
Dit betekent dat er een thema op een presentatie moet kunnen worden vastgelegd. Gevolg hiervan is ook dat kleur moet worden vastgelegd op het thema

Verder moet er een methode komen om binnen een presentatie een thema te kunnen vastleggen die onafhankelijk is van eventuele metadata die is vastgelegd om de verschillende thema's te registreren. Hierdoor kan een bestand worden gelezen zonder dat de metadata van de thema's aanwezig is.

Hier is een ontwerp beslissing die met name met de scope te maken heeft. Er kan voor worden gekozen een referentie naar het thema te leggen of de inhoud van een thema kan worden opgeslagen in het bestand. In het geval we het thema niet meenemen in het bestand, moet het thema bestand altijd op een andere locatie beschikbaar zijn en anders zou het thema niet weer worden gegeven. Op het moment dat de informatie van het specifieke thema wordt opgeslagen in de presentatie zal deze op elke lokatie te bekijken zijn. Toelichting op de keuze zal gedaan worden bij de keuzen.

\section{Ontwerp}

\includegraphics[width=14cm]{design.png}

\section{Keuzen}



\section{Sourcecode}

\end{document}
